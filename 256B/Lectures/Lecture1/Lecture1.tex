\documentclass{article}

\usepackage[english]{babel}
\usepackage{amsmath}
\usepackage{amssymb}
\usepackage{amsthm}
\usepackage[letterpaper,top=2cm,bottom=2cm,left=3cm,right=3cm,marginparwidth=1.75cm]{geometry}
\usepackage{graphicx}
\usepackage[colorlinks=true, allcolors=blue]{hyperref}
\usepackage{fancyhdr}
\usepackage{tikz}
\usetikzlibrary{matrix}
\usepackage[most]{tcolorbox}
\usepackage{hyperref}
\usepackage{array}
\usepackage{colonequals}

\theoremstyle{definition}
\newtheorem{example}{Example}[section]

\theoremstyle{definition}
\newtheorem{definition}{Definition}[section]

\theoremstyle{remark}
\newtheorem*{remark}{Remark}

\newtcbtheorem{theo}%
	{Theorem}{colback = blue!5!white,colframe = blue!50!black!50!}{theorem}
	
\newtcbtheorem{lemm}%
	{Lemma}{colback = green!5!white,colframe = green!50!black!50!}{lemma}

\newtcbtheorem{clai}%
{Claim}{colback = blue!5!white,colframe = blue!50!black!50!}{claim}

\newtcbtheorem{coroll}%
	{Corollary}{colback = purple!5!white,colframe = purple!50!black!50!}{corollary}    

\newtheorem{theorem}{Theorem}[section]
\newtheorem{corollary}{Corollary}[theorem]
\newtheorem{lemma}[theorem]{Lemma}
\newtheorem{claim}[theorem]{Claim}

\pagestyle{fancy}
\newcommand\size{1}% distance of nodes from center
\newcommand{\R}{\mathbb{R}}
\newcommand{\C}{\mathbb{C}}
\newcommand{\Z}{\mathbb{Z}}
\newcommand{\N}{\mathbb{N}}
\newcommand{\Q}{\mathbb{Q}}
\newcommand{\mb}[1]{\mathbb{#1}}
\newcommand{\mc}[1]{\mathcal{#1}}
\newcommand{\mk}[1]{\mathfrak{#1}}
\newcommand{\un}{\cup}
\newcommand{\ic}{\cap}
\begin{document}
We will more or less begin where Peter left off, but we might be able to have some leeway.
Emphasis will shift to examples, but we will discuss sheaf cohomology and derived categories.\\

First we will discuss how classical algebraic varieties are schemes, in particular reduced schemes of finite type over an algebraically field.

\textbf{Classical Varieties: } We always start with an algebraically closed field $k$, and consider the affine space $k^n$. (no need for vector space structure).
A \textbf{affine variety} is some $X \subset K^n$ which is the solution set of some polynomial equations. \\

We now consider some examples.
For $y = x^2$,  which we think with coordinates in $x,y$, the normal picture is a $"\R"^2$ cartoon, so we ought to consider $\C$.
The complex circle $x^2 + y^2 = 1$, but for $x^2 + y^2 = -1$ the set is empty over $\R$, but not $\C$.
In $\C^2$ relpacing $(x,y) \to (ix, iy)$ we swap these solution sets.
Rewriting equations we may think of any affine variety as the zero locus $V(F)$ for some family of polynomials $F = \{f_1, f_2, \dots\} \subset k[x_1, \dots, x_n]$.
Since $k$ is infinite (as it is algebraically closed), it is determined by the function 
\[f: k^n \to k\]
which is not necessarily true for finite fields. \\

\textbf{Observation 1: } For $F = \{f_1, f_2, \dots\}$ adding any polynomial in the ideal generated by it does not affect $V(F)$.
In other words 
\[V(F) = V(I)\]
where $I$ is the ideal of $k[x]$ generated by $F$.
Since $k$, and hence $k[x_1, \dots, x_n]$ are noetherian we may assume $F$ finite.\\

On the other hand, given any $Z \subset k^n$ we may define
\[I(Z) = \{f| f(p) = 0 \forall p \in Z\}\]
which is automatically an ideal.
Given $X = V(I)$, let $J = I(X)$, we clearly have 
\[X \subset V(J)\]
by definition of $J$.
Also
\[I \subset J \Longleftrightarrow V(J) \subset V(I) = X\]
so 
\[X = V(J(X))\]
However, while $V((x)) = V((x^2))$ but $(x) \neq (x^2)$.\\

\textbf{Zariski Topology}: the original one we shall define, as opposed to the scheme topology with the same name.
The closed subsets are going to be precisely the closed sets defined as $V(I)$ for ideals $I \subset k[x_1, \dots, x_n]$.
These define a topology as 
\[V((1)) = 0, \quad V((0)) = k^n\]
\[X_{\alpha} = V(I_{\alpha}) = V( \sum_{\alpha} I_{\alpha})\]
\[X_1 \un X_2 = V(I_1 I_2) = V(I_1 \ic U_2)\]
hence these closed subsets define a topology on $k^n$ which we shall call the \textbf{Zariski topology}.
$V(F)$ shall be called closed subvarieties.
Since $I_1 \ic I_2 \subset I_1, I_2$ we automatically have
\[V(I_1 \ic U_2) \supset X_1 \in X_2\]
but the latter equality is not obvious, and not even true for an infinite set.
Since if $fg(p) = 0$ implies $f(p) = 0$ or $g(p) = 0$ this gives us that 
\[V(F_1 F_2) \subset V(F_1) \un V(F_2)\]
putting it all together, since $I_1I_2 \subset I_1 \ic U_2$, so
\[V(I_1 \ic U_2) \subset V(I_1 I_2) \subset X_1 \un X_2\]
hence they are equal.
The same argument tells us that 
\[I(\{p\})\]
 is a prime ideal for a point $p \in k^n$.\\
 For any subset $X \subset k^n$ and $X = V(I)$ a closed subset, what is $V(I(Z))$.
 It clearly contains $Z$ by definition of the ideal, and simple arguments gives us that 
 \[V(I(Z)) = \overline{Z}\]
 the closure in the Zariski topology.
 We now have a correspondence 
 \[\{\text{closed} X \subset k^n\} \{\text{ideals of the form } I(Z)\}\]
Which ideals correspond to those of the form $I(V(J))$? 
By Hilbert's Nullstellenstatz we have 
\[I(V(J)) = \sqrt{J}\]
where $\sqrt{J}$ is the radical of $J$. 
Keyword: jacobson schemes, this is the crux of Hilbert's Nullstellenstatz.


\end{document}
