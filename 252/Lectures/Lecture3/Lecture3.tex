\documentclass{article}

\usepackage[english]{babel}
\usepackage{amsmath}
\usepackage{amssymb}
\usepackage{amsthm}
\usepackage[letterpaper,top=2cm,bottom=2cm,left=3cm,right=3cm,marginparwidth=1.75cm]{geometry}
\usepackage{graphicx}
\usepackage[colorlinks=true, allcolors=blue]{hyperref}
\usepackage{fancyhdr}
\usepackage{tikz}
\usetikzlibrary{decorations.markings,calc}
\usepackage{tikz-cd}
\usetikzlibrary{matrix}
\usepackage[most]{tcolorbox}
\usepackage{hyperref}
\usepackage{array}
\usepackage{colonequals}
\usepackage{todonotes}
\usepackage{theoremref}

\font\maljapanese=dmjhira at 2.5ex
\newcommand{\yo}{\textrm{\!\maljapanese\char"48}}

\newtheorem{theorem}{Theorem}[section]

\theoremstyle{definition}
\newtheorem{definition2}[theorem]{Definition}
\newtheorem{lemma}[theorem]{Lemma}
\newtheorem{corollary}[theorem]{Corollary}
\newtheorem{definition}[theorem]{Definition}
\newtheorem{example}[theorem]{Example}
\newtheorem{examples}[theorem]{Examples}


\theoremstyle{remark}
\newtheorem*{remark}{Remark}

\theoremstyle{plain}
\newtheorem{proposition}[theorem]{Proposition}
\newtheorem{conjecture}[theorem]{Conjecture}


\newcommand{\R}{\mathbb{R}}
\newcommand{\C}{\mathbb{C}}
\newcommand{\Z}{\mathbb{Z}}
\newcommand{\N}{\mathbb{N}}
\newcommand{\Q}{\mathbb{Q}}
\newcommand{\mb}[1]{\mathbb{#1}}
\newcommand{\mc}[1]{\mathcal{#1}}
\newcommand{\mk}[1]{\mathfrak{#1}}
\newcommand{\un}{\cup}
\newcommand{\ic}{\cap}
\pagestyle{fancy}
\newcommand\size{1}% distance of nodes from center

\usepackage{microtype}

\usepackage{caption}
\captionsetup[figure]{labelformat=empty}%

\begin{document}

\textbf{Today:} We prove Schur's Lemma as well as the Artin-Wedderburn Theorem.

\begin{theorem}[Schur's Lemma]
	Let $A$ be a ring with unit, and $S_1, S_2$ simple $A-$modules, then $\text{Hom}_A(S_1, S_2) = 0$ when $S_1$ and $S_2$ are not isomorphic, and otherwise $\text{End}_A(S_1)$ is a division ring.
\end{theorem}
\begin{proof}
	This one is intuitive, I doubt you'll forget bu its also in the textbook.
\end{proof}

\begin{corollary}
	If $A$ is a finite dimensional $k$-algebra for $k$ an algebraically closed field and $S$ a simple $A$-module then $\text{End}_k(S) = k$.
\end{corollary}
\begin{proof}
	iAlso in textbook, use the fact that every linear operator $V$ of vector spaces when $k$ is algebraically closed.
	If this eigenvalue is $\lambda$, the fact that $\text{End}_k(S)$ is a division ring from the first theorem tells us that $V - \lambda I = 0$, and hence all such endomorphisms are scaling by $k$.
\end{proof}

\begin{corollary}
	If $G$ is a finite abelian group, $k$ algebraically closed, then any irreducible representation of $G$ over $k$ is one-dimensional.
\end{corollary}

One question you may have asked in light of Maschke's Theorem is for what rings is it true that all modules over it are semisimple?
In light of this we consider the following definition.

\begin{definition}
	A ring $A$ is \textit{semisimple} if every $A$-moduel is semisimple.
\end{definition}

In particular $A_A$ ($A$ viewed as a left module over itself, in textbook $A$ goes on the left) is also semi-simple.
We claim the converse is also true.
Indeed, if $A_A$ is semisimple then so is ever free module, and hence every module being the quotient of a free module.
For example, from the previous lecture this implies that a division ring is semisimple.

\begin{definition}
	$M_n(D)$ is the ring of $n \times n$ matrices with coefficients in $D$.
\end{definition}

In the exercises of textbook, we show that there exists only one simple module up to isomorphism, which may be thought of as columns of the matrices.

\begin{theorem}[Artin-Wedderburn]
	Any semi-simple ring $A$ is isomorphic to the direct sum $M_{n_1}(D_1) \oplus \dots \oplus M_{n_m}(D_m)$ for some division rings $D_1, \dots, D_{m}$.
	If we also assume that $A$ is a finite dimensional $k$-algebra for $k$ algebraically closed, we may additionalyl assume that $D_1 = \dots = D_m = k$.
\end{theorem}

First note that this implies that $A_A$ decomposes into simple submodules, these by definition are minimal ideals, of which there must be finitely many if $A$ has unity, from the same argument that affine schemes are quasi-compact.
To prove Artin-Wedderburn we first state a lemma.

\begin{lemma}
	If $A$ is a ring with unity then $\text{End}_{A}(A_A) \cong \text{A}^{\text{op}}$
\end{lemma}

\begin{proof}
	There are clear maps sending an endomorphism $\phi$ to $\phi(1)$ and to an element $a \in A$ the unique endomorphism where $\phi(1) = a$, it is not hard to check these define our desired isomorphisms.
\end{proof}

We now prove Artin-Wedderburn.
If $A$ is semi-simple, let 
\[ \text{End}_A(A_A) = \text{End}_A(\bigoplus_{i=1}^n S_i^{n_i})\]
for $S_i$ simple, then by Schur's Lemma, we may ignore Homs from nonisomorphic simple modules so the above is isomorphic to 
\[\bigoplus_{i=1}^n \text{End}_A(S_i^{n_i})\]
and since $A$-modules is abelian, it is true that  $\text{End}_A(S^n) \cong M_n(\text{End}_A(S))$, a useful exercise is to construct this isomorphism.\\
Now if $D_i = \text{End}_A(S_i)$ we have from our work that 
\[A^{\text{op}} \cong \bigoplus_{i =1}^n M_n(D
_i)\]
now the transposition operation give us a natural isomorphism $M_n(D_i)^{\text{op}} \cong M_n(D_i^{\text{op}})$
hence our theorem is proved by letting $D_i = \text{End}(S_i)^{\text{op}}$, the comment for $k$ algebraically closed follows from the corollary at the beginning of lecture.
\end{document}
