\documentclass{article}

\usepackage[english]{babel}
\usepackage{amsmath}
\usepackage{amssymb}
\usepackage{amsthm}
\usepackage[letterpaper,top=2cm,bottom=2cm,left=3cm,right=3cm,marginparwidth=1.75cm]{geometry}
\usepackage{graphicx}
\usepackage[colorlinks=true, allcolors=blue]{hyperref}
\usepackage{fancyhdr}
\usepackage{tikz}
\usetikzlibrary{decorations.markings,calc}
\usepackage{tikz-cd}
\usetikzlibrary{matrix}
\usepackage[most]{tcolorbox}
\usepackage{hyperref}
\usepackage{array}
\usepackage{colonequals}
\usepackage{todonotes}
\usepackage{theoremref}

\font\maljapanese=dmjhira at 2.5ex
\newcommand{\yo}{\textrm{\!\maljapanese\char"48}}

\newtheorem{theorem}{Theorem}[section]

\theoremstyle{definition}
\newtheorem{definition2}[theorem]{Definition}
\newtheorem{lemma}[theorem]{Lemma}
\newtheorem{corollary}[theorem]{Corollary}
\newtheorem{definition}[theorem]{Definition}
\newtheorem{example}[theorem]{Example}
\newtheorem{examples}[theorem]{Examples}


\theoremstyle{remark}
\newtheorem*{remark}{Remark}

\theoremstyle{plain}
\newtheorem{proposition}[theorem]{Proposition}
\newtheorem{conjecture}[theorem]{Conjecture}


\newcommand{\R}{\mathbb{R}}
\newcommand{\C}{\mathbb{C}}
\newcommand{\Z}{\mathbb{Z}}
\newcommand{\N}{\mathbb{N}}
\newcommand{\Q}{\mathbb{Q}}
\newcommand{\mb}[1]{\mathbb{#1}}
\newcommand{\mc}[1]{\mathcal{#1}}
\newcommand{\mk}[1]{\mathfrak{#1}}
\newcommand{\un}{\cup}
\newcommand{\ic}{\cap}
\pagestyle{fancy}
\newcommand\size{1}% distance of nodes from center

\usepackage{microtype}

\usepackage{caption}
\captionsetup[figure]{labelformat=empty}%

\begin{document}

The letter $A$ will denote an arbitrary ring, then a nonzero module $M$ is a simple module if it does not have any proper non-zero submodules.
Note that
\[ \text{$M$ is simple} \Longrightarrow \text{ $M$ is generated by 1 element}\]
hence it is cyclic so we must have
\[M \cong A/I\]
for $I$ a maximal ideal.

\begin{proposition}
	Suppose that $V$ is a simple representation of a finite group $G$ over  a field $F$. Then $\text{dim}(V) \leq |G|$. 
\end{proposition}
 \begin{proof}
	 Let $A = FG$, then $\text{dim}_F A = |G|$, so $V = FG/I$ so $\text{dim}_F V \leq |G|$.
 \end{proof}
 
 Now suppose that $G = C_{p^n}$ a cyclic field group and $\text{char} F = p$.
 Then any irreducible representation of $G$ over $F$ is trivial. 
 Indeed, if $C_{p^n} = \langle g \rangle$, then $FG = F[g]/(g^{p^n}-1)$.
 For $x = g-1$ this is $F[x]/(x^{p^n})$ is a local ring with unique maixmal ideal $(x)$, so the only representation is $F[x]/(x) = F$.

 \subsection{Semisimple modules in general}

 \begin{proposition}
	 The following conditions on an $A$-module $V$ are equivalent.\\
	 \indent (1) $V$ is a sum of all its simple submodules.\\
	 \indent (2) $V = \bigoplus_i S_i$ for simple submodules $S_i$.\\
	 \indent (3) Every submodule $W \subset V$ is a direct summand of $V$, i.e. $V = W \oplus W'$ for some submodule $W'$.
	 An $A$-module is called \textit{semisimple} if it satisfies any of the above equivalent conditions.
 \end{proposition}
 
 \begin{proof}
 This is proved in the textbook in the finite case, it holds in general by an application of Zorn's Lemma.
 We give a sketch of the proof.\\

 $(1) \Longrightarrow (2):$ If $\{S_j\}_{j \in J}$ is all simple submodules, then by Zorn's there exists some $I \subset J$ such that 
 \[ \sum_{i \in I} S_i = \bigoplus_{i \in I} S_i\]
 we can check that $V = \bigoplus_{i \in I} S_i$.\\

 $(2) \Longrightarrow (3):$ For a submodule $W$, choose a maximal subset $K \subset I$ such that $W \ic (\bigoplus_{i \in I} S_i) = \{0\}$ then $\bigoplus_{j \in I/K} S_j = W'$ which is readily checked using properties of simple submodules.\\

 $(3) \Longrightarrow (1)$: See the textbook.\\

 We first give a definition.
\begin{definition}
	The \textit{socle} of $V$ is the sum of all simple submodules of $V$, we will also denote this as \textbf{Soc}($V$).
\end{definition}
Assuming (3) holds, we first claim that Soc($V$) is nonzero.
For $v \in V, v \neq 0$, there exists a maximal $M \subset Av$.
Then $V = Av \oplus N$ then $M \oplus N \subset V$ is maximal, hence $S = W/M \oplus N$ is simple.
Therefore $V = M \oplus N \oplus S$ by property 3, and another use gives us that $\text{Soc}(V) = V$ as we wanted to show.

\end{proof}

\begin{proposition}
	A quotient and submodule of a semisimple module is semi-simple.\end{proposition}

\begin{proof}
	The first is readily verified using the equivalent properties proven above. 
\end{proof}

As an example, if $D$ is a division ring then any $D$-module is semisimple over itself.

Go to a representation of $G$, a semisimple $FG$-module is called a \textit{completely reducible} representation of $G$.
We shall now prove the firsrt main theorem of representation theorem

\begin{theorem}[Maschke's Theorem]
	Let $G$ be a finite group and $F$ be a field such that the characteristic of $F$ does not divide $|G|$.
	Then any representation of $G$ over $F$ is completely irreducible.
\end{theorem}
\begin{proof}
	We verify property (3).
	Let $V$ be a $FG$ module, and $W$ any invariant subspace (subrepresentation): we wish to find a complementary subspace $W'$ such that $V = W \oplus W'$.
	There always exists such a subspace $W''$ (take the kernel of a projection $\pi$ of $V$ onto $W$).
	However $W''$ is not necessarily $G$-invariant.
	Define 
	\[ \overline{\pi} = \frac{1}{{G}} \sum_{g \in G} g \pi g^{-1}\]
which is well defined as $|G|$ is assumed invertible in $F$. \\

We first show that $\overline{\pi} \in \text{End}_{FG}(V)$. Indeed, for any $h \in G$ $h \overline{\pi} h^{-1} = \overline{\pi}$ by a simple computation.
Also note that $\overline{\pi}(V)$ is also a projection onto $W$ since $W$ is invariant under $FG$ and $\pi$ is the identity rericteded to $W$.
We shall show that $W'' = \text{ker}(\overline{\pi})$ is our desired subspace. 
Indeed, if $\overline{\pi}w = 0$ then $\overline{\pi}(gw) = g \overline{\pi} w = 0$.
\end{proof}

\begin{example}
	Take $G = S_3$, consider the permutation representation in $F^3 = V$.
	Take $W = \{a_1e_1  + a_2e_2 + a_3e_3\}$ such that $a_1 + a_2 + a_3 = 0$.
	If $\text{char} F \neq 3$, then $W' = a(e_1 + e_2 + e_3)$ is complementary.
	However, if $\text{char} F = 3$ then $W' \subset W$ so it doesn't hold.
	Hence $\text{Soc}(V) = V$ if $\text{char}(F) \neq 3$ and $W'$ if $\text{char}(F) = 3$ .
\end{example}

We introduce the next result but don't prove it, you may recognize it as a general form of a theorem often encountered in group theory.

\begin{theorem}[Jordan Holder]
	Let $V$ be an $A$-module. $V$ is of finite length if there exists a finite filtration of submodules 
	\[0 = V_1 \subset V_1 \subset \dots \subset V_e = V\]
	of submodules such that $V_i/V_{i-1}$ is simple. 
	If $V$ is of finite length, then any other such filtration has the same length and simple quotients, though not necessarily in the same order.
	These $S_i$'s are called \textit{simple constituents} of $V$.
\end{theorem}

\end{document}
